%	This is written by Zhiyang Ong as a template for typesetting in LaTeX.

%	The MIT License (MIT)

%	Copyright (c) <2014> <Zhiyang Ong>

%	Permission is hereby granted, free of charge, to any person obtaining a copy of this software and associated documentation files (the "Software"), to deal in the Software without restriction, including without limitation the rights to use, copy, modify, merge, publish, distribute, sublicense, and/or sell copies of the Software, and to permit persons to whom the Software is furnished to do so, subject to the following conditions:

%	The above copyright notice and this permission notice shall be included in all copies or substantial portions of the Software.

%	THE SOFTWARE IS PROVIDED "AS IS", WITHOUT WARRANTY OF ANY KIND, EXPRESS OR IMPLIED, INCLUDING BUT NOT LIMITED TO THE WARRANTIES OF MERCHANTABILITY, FITNESS FOR A PARTICULAR PURPOSE AND NONINFRINGEMENT. IN NO EVENT SHALL THE AUTHORS OR COPYRIGHT HOLDERS BE LIABLE FOR ANY CLAIM, DAMAGES OR OTHER LIABILITY, WHETHER IN AN ACTION OF CONTRACT, TORT OR OTHERWISE, ARISING FROM, OUT OF OR IN CONNECTION WITH THE SOFTWARE OR THE USE OR OTHER DEALINGS IN THE SOFTWARE.

%	Email address: echo "cukj -wb- 23wU4X5M589 TROJANS cqkH wiuz2y 0f Mw Stanford" | awk '{ sub("23wU4X5M589","F.d_c_b. ") sub("Stanford","d0mA1n"); print $5, $2, $8; for (i=1; i<=1; i++) print "6\b"; print $9, $7, $6 }' | sed y/kqcbuHwM62z/gnotrzadqmC/ | tr 'q' ' ' | tr -d [:cntrl:] | tr -d 'ir' | tr y "\n"

%%%%%%%%%%%%%%%%%%%%%%%%%%%%%%%%%%%%%%%%%%%%%%



%%%%%%%%%%%%%%%%%%%%%%%%%%%%%%%%%%%%%%%%%%%
\chapter{Text}
\label{chp:Text}

There are a significant amount of references for helping people to learn \LaTeX \cite{Voss2011,vanDongen2012,Syropoulos2003,Raymond2004,Mittelbach2004,Lamport1994,Krishnan2003,Krantz2001,Kottwitz2011,Koranne2011,Kopka2004,Knuth1999,Hoenig1998,Higham1998,Haralambous2007,Griffiths1997,Gratzer2007,Goossens2007,Goossens1999,Goossens1997,Diller1999,Bindner2011,Berry2009,UITCambridge2011,Scharrer2011,Pakin2008,Cormen2010,Syropoulos2004,Hamalainen2006} and related information/technologies. \\


In this chapter, I will provide some templates for referencing, templates for {\sc Bib}\TeX\ entries, indicate some common \LaTeX\ symbols, usage of colors in \LaTeX, and miscellaneous details. \\


Random macros from my \LaTeX-specific IDE (or text editor): \vspace{-0.3cm}
\begin{enumerate} \itemsep -4pt
\item $\backslash\backslash\ \backslash$rule\{6in\}\{.1pt\}			%	\\ \rule{6in}{.1pt}
\item \href{mailto:emailid@domain.com}{emailid@domain.com}		%	\href{mailto:emailid@domain.com}{emailid@domain.com}
\item Begin-end constructs (i.e., $\backslash$begin and $\backslash$end) for: \vspace{-0.3cm}
	\begin{enumerate} \itemsep -2pt
	\item quotation
	\item quote
	\item verbatim
	\item verse
	\end{enumerate}
\item Types of headings: \vspace{-0.3cm}
	\begin{enumerate} \itemsep -2pt
	\item $\backslash$chapter\{\}
	\item $\backslash$paragraph\{\}
	\item $\backslash$subparagraph\{\}
	\item $\backslash$section\{\}
	\item $\backslash$subsection\{\}
	\item $\backslash$subsubsection\{\}
	\end{enumerate}
\item To add an entry into the ``Table of Contents'' without it being numbered, try the following: \vspace{-0.3cm}
	\begin{enumerate} \itemsep -2pt
	\item $\backslash$addcontentsline\{toc\}\{section\}\{BLAH\}
	\item $\backslash$section$^{\ast}$\{BLAH\}
	\end{enumerate}
\item Insert/import content from another file: $\backslash$input\{RELATIVE PATHNAME\}
\item Import \LaTeX\ packages: $\backslash$usepackage\{\}
\item $\backslash$footnote\{\}
\item $\backslash$marginpar\{\}
\item $\mathcal{C}$
\item $\mathit{C}$: Caligraphy style font.
\item \underline{This is good.}: Underline text.
\item \texttt{This is a statement.} TypeWriter.
\item \textsf{This is a statement.} Sans Serif font.
\item \textsl{This is a statement.} Slanted font.
\item \emph{This is a statement.}
\item Types of labels: \vspace{-0.3cm}
	\begin{enumerate} \itemsep -2pt
	\item ``chp:'' for chapter
	\item ``sec:'' for section
	\item ``ssec:'' for subsection
	\item ``sssec:'' for subsubsection
	\item ``fig:'' for figure
	\item ``tab:'' for table
	\item ``eqn:'' for equation
	\item ``lst:'' for code listing
	\item ``defn:'' for definition
	\item ``thrm:'' for theorem
	\item ``lem:'' for lemma
	\item ``crly:'' for corollary
	\item ``prop:'' for proposition
	\item ``prf:'' for proof
	\item ``eg:'' for example
	\item ``rem:'' for remark
	\end{enumerate}
\end{enumerate}


An enumeration of items: \vspace{-0.3cm}
\begin{enumerate} \itemsep -4pt
\item Quite sparse enumeration: \vspace{-0.3cm}
	\begin{enumerate} \itemsep -2pt
	\item Sparse enumeration: \vspace{-0.2cm}
		\begin{enumerate} \itemsep -2pt
		\item Very sparse enumeration: \vspace{-0.1cm}
			\begin{enumerate} \itemsep -1pt
			\item Very, very sparse list: \vspace{-0.1cm}
				\begin{itemize} \itemsep -1pt
				\item Blah
				\end{itemize}
			\end{enumerate}
		\end{enumerate}
	\end{enumerate}
\item 
\item 
\item Inserting a horizontal line beneath this item in the list.
\\ \rule{6in}{.1pt}
\item 
\item 
\end{enumerate}

List of items: \vspace{-0.3cm}
\begin{itemize} \itemsep -4pt
\item Blah
\end{itemize}

Description of items: \vspace{-0.3cm}
\begin{description} \itemsep -4pt
\item[Key] Sparse description: \vspace{-0.3cm}
	\begin{description} \itemsep -2pt
	\item[key] Another entry
	\end{description}
\end{description}







